
\begin{abstract}
    Los objetos compactos son remanentes de la vida luminosa de las estrellas, originados en gigantes explosiones conocidas como supernovas. En estos objetos se alcanzan densidades y campos gravitacionales tan grandes que la Relatividad General es importante para describir su estructura. Uno de los problemas abiertos de la astrofísica es encontrar la ecuación de estado de estos objetos, ya que no se conoce el comportamiento de la materia a densidades tan altas. 
    Una forma de evaluar si usando las diferentes ecuaciones de estado se obtienen modelos estelares de interés astrofísico es verificando que los modelos cumplan con algunas condiciones de aceptabilidad física.
    
    En este propuesta se propone usar un criterio introducido recientemente por Nuñez et al. \cite{Hernandez2018}, para determinar si los modelos de objetos compactos obtenidos con las ecuaciones de estado encontradas en la literatura son estables ante movimientos convectivos.
    \REMARK{Se puede pulir.}
\end{abstract}

%\begin{abstract1}
 %   The abstract should be short, stating what you did and what the most important result is.
%	Lorem ipsum dolor sit amet, consetetur sadipscing elitr, sed diam nonumy eirmod tempor invidunt ut labore et dolore magna aliquyam erat, sed diam voluptua. At vero eos et accusam et justo duo dolores et ea rebum. Stet clita kasd gubergren, no sea takimata sanctus est Lorem ipsum dolor sit amet. Lorem ipsum dolor sit amet, consetetur sadipscing elitr, sed diam nonumy eirmod tempor invidunt ut labore et dolore magna aliquyam erat, sed diam voluptua. At vero eos et accusam et justo duo dolores et ea rebum. Stet clita kasd gubergren, no sea takimata sanctus est Lorem ipsum dolor sit amet.
%\end{abstract1}