\chapter{Metodología}
\setcounter{footnote}{0}
Es importante resaltar que debido a que ecuaciones de TOV no se pueden resolver analíticamente y las ecuaciones de estado se encuentran tabuladas, la naturaleza del problema es computacional. 
Se propone como metodología para cumplir con los objetivos específicos planteados anteriormente cumplir las siguienets actividades:

\begin{enumerate}
    \item Reunir ecuaciones de estado para la materia densa tabuladas de la literatura, necesarias para resolver las ecuaciones de TOV.
    \item Interpolar las ecuaciones de estado manera óptima, explorando las distintas posibilidades.
    \item Resolver las ecuaciones de TOV numéricamente para las diferentes ecuaciones de estado para obtener modelos de estrellas de neutrones estáticas. Se usará en lo posible el ecosistema de software de código abierto para cómputo científico (SciPy\footnote{\url{https://www.scipy.org}}) de Python.
    \item Obtener las derivadas y segundas derivadas de los perfiles de densidad $\rho(r)$ y presión $P(r)$ obtenidos, lo cual permitirá evaluar la mayoría las condiciones de aceptabilidad física. 
    \item Variar la densidad central para obtener la familia de configuraciones en equilibrio asociada a la ecuación de estado y graficar las relaciones $M-R$ y $M-\rho_c$, lo cual permitirá identificar la masa máxima de la familia de soluciones y evaluar la condición C10 respectivamente.
\end{enumerate}
\section{Cronograma}
\definecolor{royalazure}{rgb}{0.0, 0.22, 0.66}
\centering
\vspace{-0.5cm}
\noindent\resizebox{0.28\textwidth}{!}{
\begin{ganttchart}[
    canvas/.append style={line width=.60pt,dotted},
    hgrid style/.style={line width=.60pt,dotted},
    vgrid={*1{ line width=.60pt,dotted}},
    title/.style={draw=none, fill=none},
    title label font=\bfseries\footnotesize,
    title label node/.append style={below=7pt},
    include title in canvas=false,
    group label font=\small\color{black},
    bar/.append style={fill=royalazure},
    group left shift=0,
    group right shift=0,
    bar height=.3,
    group peaks tip position=0,
    bar label node/.append style={left=0.6cm}
  ]{1}{4}
  \gantttitle[
    title label node/.append style={below left=7pt and -7pt}
  ]{\quad Mes:\quad 1}{1}
  \gantttitlelist{2,...,4}{1} \\ 
  \ganttbar{Actividad 1}{1}{1} \\[grid]
  \ganttbar{Actividad 2}{1}{2} \\[grid]
  \ganttbar{Actividad 3}{2}{3} \\[grid]
  \ganttbar{Actividad 4}{3}{4} \\[grid]
  \ganttbar{Actividad 5}{3}{4}
\end{ganttchart}}
\vspace{8cm}

