\chapter{Planteamiento del problema}

Los modelos estáticos son el punto de partida de un estudio dinámico de objetos compactos: los efectos de la rotación lenta se pueden introducir mediante teoría de perturbaciones \cite{Hartle1967} y la evolución térmica se puede estudiar en un régimen cuasi-estático \cite{Becerra2015Quasi-staticObjects}. Es por esto que obtener soluciones a las ecuaciones de estructura relativistas, dada una determinada ecuación de estado para la materia densa, y analizar su estabilidad \hl{ante distintas perturbaciones/con distintos criterios} es muy importante a la hora de identificar modelos \hl{realistas/satisfactorios/correctos} de objetos compactos.

En el trabajo de grado se propone solucionar las ecuaciones de TOV \eqref{dmtov} y \eqref{dptov} numéricamente para ecuaciones de estado que se encuentren en la literatura actual, y analizar su estabilidad global contra pulsaciones radiales usando el criterio estándar de Harrison-Zeldovich-Novikov y su estabilidad local contra convección adiabática usando el criterio presentando por Nuñez et al. \cite{Hernandez2018}.\TODO{Añadir más cosas? Masa-Radio!}
\REMARK{Modelos astrofísicos realistas necesitan una ecuación de estado realista.}
\REMARK{La única forma de hacerlo es numéricamente. Por la complejidad de las ecuaciones y porque las ecuaciones de estado no tienen forma analítica.}
\REMARK{Es importante analizar la estabilidad de los modelos en equilibrio obtenidos con las ecuaciones de estado numéricas.}
\REMARK{Se escogerán ecuaciones de estado en la literatura, se hallarán sus configuraciones de equilibrio (resolviendo TOV) usando Python y se evaluará su estabilidad tanto global como local (usando los criterios más simples). }
\REMARK{Se buscará que los resultados sean reproducibles y abiertos a la comunidad científica.}