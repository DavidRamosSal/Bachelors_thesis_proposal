\chapter{Planteamiento del problema}
De los modelos estáticos surgen las primeras aproximaciones al estudio dinámico de objetos compactos: los efectos de la rotación lenta se pueden introducir mediante teoría de perturbaciones \cite{Hartle1967} y la evolución térmica se puede estudiar en un régimen cuasi-estático \cite{Becerra2013Quasi-staticObjects}. Es por esto que obtener soluciones a las ecuaciones de estructura relativistas, dada una determinada ecuación de estado para la materia densa, y verificar que los modelos obtenidos cumplan las condiciones de aceptabilidad física es muy importante a la hora de desarrollar modelos dinámicos de objetos compactos.

En este trabajo de grado se propone determinar si los modelos de estrellas de neutrones obtenidos a partir de las ecuaciones de estado más relevantes encontradas en la literatura son estables ante movimientos convectivos, usando la condición C11 propuesta por Nuñez et al.
