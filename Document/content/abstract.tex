
\begin{abstract}
    Los objetos compactos son remanentes de la vida luminosa de las estrellas, originados en gigantes explosiones conocidas como supernovas. En estos objetos se alcanzan densidades y campos gravitacionales grandes donde la Relatividad General es importante para describir su estructura. La ecuación de estado de estos objetos es una de los problemas abiertos de la astrofísica, ya que no se conoce el comportamiento de la materia a densidades muy altas. 
    Una forma de evaluar si las distintas ecuaciones de estado describen correctamente estos objetos, es verificando que los modelos obtenidos con ellas cumplan con ciertas condiciones de estabilidad, acoplamiento y regularidad.
    
    En este propuesta se propone determinar si los modelos de objetos compactos obtenidos con las ecuaciones de estado encontradas en la literatura son estables ante movimientos convectivos.
\end{abstract}

%\begin{abstract1}
 %   The abstract should be short, stating what you did and what the most important result is.
%	Lorem ipsum dolor sit amet, consetetur sadipscing elitr, sed diam nonumy eirmod tempor invidunt ut labore et dolore magna aliquyam erat, sed diam voluptua. At vero eos et accusam et justo duo dolores et ea rebum. Stet clita kasd gubergren, no sea takimata sanctus est Lorem ipsum dolor sit amet. Lorem ipsum dolor sit amet, consetetur sadipscing elitr, sed diam nonumy eirmod tempor invidunt ut labore et dolore magna aliquyam erat, sed diam voluptua. At vero eos et accusam et justo duo dolores et ea rebum. Stet clita kasd gubergren, no sea takimata sanctus est Lorem ipsum dolor sit amet.
%\end{abstract1}