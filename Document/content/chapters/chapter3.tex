\chapter{Planteamiento del problema}
De los modelos estáticos surgen las primeras aproximaciones al estudio dinámico de objetos compactos: los efectos de la rotación lenta se pueden introducir mediante teoría de perturbaciones \cite{Hartle1967} y la evolución térmica se puede estudiar en un régimen cuasi-estático \cite{Becerra2013Quasi-staticObjects}. Es por esto que obtener soluciones a las ecuaciones de estructura relativistas, dada una determinada ecuación de estado para la materia densa, es muy importante a la hora de desarrollar modelos dinámicos de objetos compactos.
Ahora bien, a la hora de crear modelos estáticos surgen las preguntas: ¿Qué ecuación de estado para la materia ultradensa escoger? y  ¿existen criterios físicos para escoger entre las distintas ecuaciones de estado disponibles? 

Verificar que los modelos de objetos compactos obtenidos con las distintas ecuaciones de estado cumplan las condiciones de aceptabilidad física puede ser una manera de discernir si éstas los describen apropiadamente. En este trabajo de grado se propone determinar si los modelos de estrellas de neutrones obtenidos a partir de las ecuaciones de estado más relevantes encontradas en la literatura son estables ante movimientos convectivos, usando la condición C11 propuesta por Nuñez et al.
